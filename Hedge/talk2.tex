% $Header: /data/cvsroot/Courses/OnlineLearning/talks/talk2/talk2.tex,v 1.5 2006/01/17 08:11:25 yfreund Exp $

%\documentclass{beamer}
\documentclass[handout]{beamer}
% This file is a solution template for:

% - Giving a talk on some subject.
% - The talk is between 15min and 45min long.
% - Style is ornate.

% Copyright 2004 by Till Tantau <tantau@users.sourceforge.net>.
%
% In principle, this file can be redistributed and/or modified under
% the terms of the GNU Public License, version 2.
%
% However, this file is supposed to be a template to be modified
% for your own needs. For this reason, if you use this file as a
% template and not specifically distribute it as part of a another
% package/program, I grant the extra permission to freely copy and
% modify this file as you see fit and even to delete this copyright
% notice. 


\mode<presentation>
{
  \usetheme{Montpellier}

  %\setbeamercovered{transparent}
  % or whatever (possibly just delete it)
}

\usepackage{xmpmulti} % package that defines \multiinclude

\usepackage[english]{babel}

\usepackage[latin1]{inputenc}

\usepackage{times}
\usepackage[T1]{fontenc}
% Or whatever. Note that the encoding and the font should match. If T1
% does not look nice, try deleting the line with the fontenc.

\title[\ouralg] % (optional, use only with long paper titles)
{Exponential Weights Algorithms for Online Learning}

\author[Freund] % (optional, use only with lots of authors)
{Yoav Freund}
% - Give the names in the same order as the appear in the paper.
% - Use the \inst{?} command only if the authors have different
%   affiliation.

\institute[Universities of Somewhere and Elsewhere] % (optional, but mostly needed)

\subject{Machine Learning}
% This is only inserted into the PDF information catalog. Can be left
% out. 

% If you have a file called "university-logo-filename.xxx", where xxx
% is a graphic format that can be processed by latex or pdflatex,
% resp., then you can add a logo as follows:

% \pgfdeclareimage[height=0.5cm]{university-logo}{university-logo-filename}
% \logo{\pgfuseimage{university-logo}}



% Delete this, if you do not want the table of contents to pop up at
% the beginning of each subsection:
%% \AtBeginSubsection[]
%% {
%%   \begin{frame}<beamer>
%%     \frametitle{Outline}
%%     \tableofcontents[currentsection,currentsubsection]
%%   \end{frame}
%% }


% If you wish to uncover everything in a step-wise fashion, uncomment
% the following command: 

\beamerdefaultoverlayspecification{<+->}

\newcommand{\R}[1]{{\color{red}{#1}}}
\newcommand{\W}{\vec{W}}
\newcommand{\V}{\vec{V}}
\newcommand{\X}{\vec{X}}
\newcommand{\loss}{\vec{\ell}}
\newcommand{\HedgeLoss}{L_{\mbox{\footnotesize Hedge}}}

\newcommand{\newmcommand}[2]{\newcommand{#1}{{\ifmmode {#2}\else\mbox{${#2}$}\fi}}}
\newcommand{\renewmcommand}[2]{\renewcommand{#1}{{\ifmmode {#2}\else\mbox{${#2}$}\fi}}}
\newcommand{\newmcommandi}[2]{\newcommand{#1}[1]{{\ifmmode {#2}\else\mbox{${#2}$}\fi}}}
\newcommand{\newmcommandii}[2]{\newcommand{#1}[2]{{\ifmmode {#2}\else\mbox{${#2}$}\fi}}}
\newcommand{\newmcommandiii}[2]{\newcommand{#1}[3]{{\ifmmode {#2}\else\mbox{${#2}$}\fi}}}

\newcommand{\algfnt}{\bf}

\newmcommand{\ouralg}{{\mbox{\algfnt Hedge}({\eta})}}

\newmcommand{\iter}{T}

\newfont{\cmmib}{cmmib10}
\newcommand{\boldell}{{\mbox{\cmmib \symbol{'140}}}}


\newmcommandi{\costvec}{{\boldell}^{#1}}
\newmcommandii{\cost}{{\ell}^{#1}_{#2}}

\newmcommandi{\rd}{\tilde{#1}}

\newmcommandi{\distvec}{{\bf p}^{#1}}
\newmcommandi{\rddistvec}{\rd{\bf p}^{#1}}
\newmcommandii{\dist}{{p}^{#1}_{#2}}
\newmcommandii{\rddist}{\rd{p}^{#1}_{#2}}

\newmcommandi{\bdistvec}{{\bf q}^{#1}}
\newmcommandii{\bdist}{{q}^{#1}_{#2}}

\newmcommandi{\wtvec}{{\bf w}^{#1}}
\newmcommandi{\rdwtvec}{\rd{\bf w}^{#1}}
\newmcommandii{\wt}{{w}^{#1}_{#2}}
\newmcommandii{\rdwt}{\rd{w}^{#1}_{#2}}


\newcommand{\Nweight}[2]{V_{#1}^{#2}}	%the normalized weight
\newcommand{\dweight}[2]{w^{#2}(#1)} % initial density measure
\newcommand{\TEloss}[1]{L_{#1}}	%total loss of expert i
\newcommand{\BEloss}{L_{\min}}	%total loss of the best expert
\newcommand{\TAloss}{L_A}	%total loss of algorithm
\newcommand{\weight}[2]{W_{#1}^{#2}} % weight assigned to expert
\newcommand{\btheta}{\hat{\theta}}

\newcommand{\R}[1]{{\color{red}{#1}}}
\newcommand{\B}[1]{{\color{blue}{#1}}}
\newcommand{\RM}[1]{{\color{red}{$#1$}}}


%BANDITS
\newcommand{\Aplay}{{\bf Hedge}}
\newcommand{\Aest}{{\bf Exp3}}
\newcommand{\Aesthp}{{\bf Exp3.P}}
\newcommand{\Aestg}{{\bf Exp3.P.1}}
\newcommand{\Aests}{{\bf Exp3.S}}
\newcommand{\Aessg}{{\bf Exp3.S.1}}
\newcommand{\Astrat}{{\bf Exp4}}
\newcommand{\Abound}{{\bf Exp3.1}}
\newcommand{\Gbest}{G_{\rm max}}

\newcommand{\defeq}{\stackrel{\rm def}{=}}
\newcommand{\compl}{\mbox{\sc h}}
\newcommand{\theset}[2]{\{ {#1} \,:\, {#2} \}}

\newmcommandii{\stratv}{\mbox{\boldmath $\xi$}^{#1}({#2})}

%Games paper
\newmcommand{\M}{\bf M}
\newmcommand{\dM}{\M'}
\newmcommand{\Row}{\bf R}
\newmcommand{\dRow}{\R'}
\newmcommand{\C}{\bf C}
\newmcommand{\dC}{\C'}
\newmcommand{\D}{D}
\renewmcommand{\P}{\bf P}
\newmcommand{\Q}{\bf Q}
\newmcommand{\Dt}{\D_t}
\newmcommand{\Pt}{\P_t}
\newmcommand{\Qt}{\Q_t}
\newmcommand{\Pstar}{\P^*}	% the min/max optimal mixed strategy
\newmcommand{\Pref}{\tilde{\P}}	% a reference mixed strategy (not
				% necessarily min/max)
\newmcommand{\Qstar}{\Q^*}
\newmcommand{\Pa}{\overline{\P}}
\newmcommand{\Qa}{\overline{\Q}}
\newmcommand{\Qh}{\hat{\Q}}
\newmcommandi{\trans}{{#1}^{\rm T}}
\newmcommand{\mhx}{\M(h,x)}
\newmcommand{\mxh}{\dM(x,h)}
\newmcommand{\mpq}{\M(\P,\Q)}
\newmcommand{\mpsq}{\M(\Pstar,\Q)}
\newmcommand{\mpsqt}{\M(\Pstar,\Qt)}
\newmcommand{\mptqt}{\M(\Pt,\Qt)}
\newmcommand{\mptt}{\M(\Pt,t)}
\newmcommand{\mptq}{\M(\Pt,\Q)}
\newmcommand{\mpqt}{\M(\P,\Qt)}
\newcommand{\minp}{\min_{\P}}
\newcommand{\maxq}{\max_{\Q}}
\newcommand{\RE}[2]{{\rm RE}\left( {#1} \; \parallel \; {#2} \right) }

\newmcommand{\sumt}{\sum_{t=1}^T}
\newmcommand{\sumin}{\sum_{i=1}^n}
\newmcommand{\delt}{\Delta_{T,n}}
\newcommand{\nextline}{\vspace{0.2cm}\\}   % a little space for equation arrays

\newcommand{\lwalg}{\mbox{\rm MW}}
\newcommand{\lwalgvar}{\mbox{\rm vMW}}

%%
\newcommand{\E}{\mbox{\rm\bf E}}
\newcommand{\p}[2]{p_{#1}(#2)}
\newcommand{\q}[2]{q_{#1}(#2)}
\newcommand{\x}[2]{x_{#1}({#2})}
\newmcommand{\bx}{\mbox{\boldmath$x$}}
\newmcommandi{\xv}{\bx({#1})}
\newmcommand{\xvt}{\xv{t}}
%\newcommand{\w}[2]{w_{#1}({#2})} replaced by \wt, but remember to switch order of parameters i and t
\renewcommand{\i}[1]{i_{#1}}
\newcommand{\hx}[2]{\hat{x}_{#1}(#2)}
\newcommand{\hxit}{\hx{\i{t}}{t}}
\newcommand{\pit}{\p{\i{t}}{t}}
\newcommand{\xit}{\x{\i{t}}{t}}
\newcommand{\expb}[1]{\exp\left(#1\right)}

\newcommand{\vp}{{\mathbf p}}
\newcommand{\vu}{{\mathbf u}}
\newcommand{\vv}{{\mathbf v}}
\newcommand{\vx}{{\mathbf x}}
\newcommand{\vy}{{\mathbf y}}

\newcommand{\HedgeLoss}{L_{\mbox{\footnotesize Hedge}}}

\newcommand{\W}{\vec{W}}
\newcommand{\V}{\vec{V}}
\newcommand{\X}{\vec{X}}
\newcommand{\vb}{\vec{b}}
%\newcommand{\loss}{\vec{\ell}}
\newcommand{\loss}{L}
%\newcommand{\elloss}[2]{\ell_{#1}^{#2}} %loss of expert i at time t
\newcommand{\elloss}[2]{\ell_{#2}\left( #1 \right)} %loss of expert i at time t
\newcommand{\w}[1]{\makebox[12pt]{{#1}}}
\newcommand{\Rps}{\mbox{\tt R}}
\newcommand{\rPs}{\mbox{\tt P}}
\newcommand{\rpS}{\mbox{\tt S}}
\newcommand{\rpstie}{\w{$\frac{1}{2}$}}
\newcommand{\rpswin}{\w{$0$}}
\newcommand{\rpsloss}{\w{$1$}}

\newmcommand{\decspace}{\Delta}
\newmcommand{\decsym}{\delta}
\newmcommandi{\dec}{\decsym^{#1}}
\newmcommand{\decdistsym}{\cal D}
\newmcommandi{\decdist}{{\decdistsym}^{#1}}

\newmcommand{\simpdistspace}{{\bf \cal S}}
\newmcommand{\domset}{{\rm dom}(\decdistsym)}

\newmcommand{\expdistsym}{{\cal E}}
\newmcommandii{\expdist}{{\expdistsym}^{#1}_{#2}}
\newmcommand{\expdecsym}{{\varepsilon}}
\newmcommandii{\expdec}{\expdecsym^{#1}_{#2}}

\newmcommand{\outspace}{\Omega}
\newmcommand{\outsym}{\omega}
\newmcommandi{\out}{\outsym^{#1}}

%\newmcommandii{\Dkl}{D_{\mbox{kl}}\paren{#1||#2}}
\newmcommandii{\Dkl}{{\rm {KL}}\paren{{#1}\;||\;{#2}}}

\newmcommandi{\sumwts}{\sum_{i=1}^N \wt{#1}{i}}

\newmcommand{\lossalg}{L_A}
\newmcommand{\lossouralg}{{L_{\mbox{\scriptsize\algfnt Hedge}(\eta)}}}
\newmcommand{\lossS}{{L_{\mbox{\scriptsize\algfnt S}}}}
\newmcommandi{\lossi}{L_{#1}}
\newmcommandii{\lossit}{L_{#1}^{#2}}

\newmcommandi{\upbnd}{\tilde{#1}}

\newcommand{\angles}[1]{{\left\langle {#1} \right\rangle}}
\newcommand{\paren}[1]{{\left( {#1} \right)}}
\newcommand{\brac}[1]{{\left[ {#1} \right]}}
\newcommand{\braces}[1]{{\left\{ {#1} \right\}}}

\newcommand{\abs}[1]{{\left| {#1} \right|}}
\newcommand{\ceiling}[1]{{\left\lceil {#1} \right\rceil}}

\newfont{\msym}{msbm10}
\newcommand{\real}{\mbox{\msym R}}

\newmcommand{\updatefcn}{U_\eta}

%% \newtheorem{theorem}{Theorem}	
%% \newtheorem{lemma}[theorem]{Lemma}
%% \newtheorem{corollary}[theorem]{Corollary}
%% \newtheorem{definition}{Definition}

%\newcommand{\proof}{\noindent{\bf Proof:} }
%\newcommand{\example}[1]{{\em Example #1.} }
%\newcommand{\qed}{\rule{0.7em}{0.7em}}

\newcommand{\WeakAlg}{\mbox{\algfnt WeakLearn}}
\newcommand{\Boost}{\mbox{\algfnt AdaBoost}}
\newcommand{\EX}{\mbox{\bf EX}}
\newmcommand{\hf}{h_{{f}}}
\newmcommand{\rdhf}{\rd{h}_{{f}}}
\newmcommand{\hfT}{h^T_{{f}}}
\newmcommand{\ranh}{{b}}

\newmcommand{\conclass}{{\cal C}}

\newmcommand{\badvec}{{\bf b}}
\newmcommandi{\bad}{{b}_{#1}}

%%%%%%%% New commands defined for the game-playing paper

\newmcommand{\hedge}{\algfnt Hedge}
\newmcommand{\play}{\algfnt Play}
\newmcommandi{\Glossvec}{{\bg y}^{#1}}
\newmcommandii{\Gloss}{{y}^{#1}_{#2}}
%\newmcommandi{\action}{{I}_{#1}}
\newmcommandi{\Gdistvec}{{\bf \tilde{p}}^{#1}}
\newmcommandii{\Gdist}{{\teilde{p}}^{#1}_{#2}}

%%%%%%%%%%%%%%%%%%%%%%%%%%%%%%%%%%%%%%%%%%%%%%%%%%%%%
\newmcommand{\Idistvec}{{D}}
\newmcommandi{\Idist}{\Idistvec({#1})}
\newmcommand{\Idistt}{\Idistvec_t}

\newmcommand{\Xdist}{{\cal P}}
\newmcommand{\emp}{\hat{\epsilon}}

\newmcommand{\classpc}{Y}
\newmcommand{\numclass}{k}
\newmcommandii{\prob}{\mbox{\rm Pr}_{#1}\left[{#2}\right]}
\newmcommandii{\exval}{\mbox{\rm E}_{#1}\left[{#2}\right]}

%\usepackage{amsmath}
\DeclareMathOperator*{\argmax}{argmax} % thin space, limits underneath in displays
\DeclareMathOperator*{\argmin}{argmin} 

\newcommand{\RR}{\mathbb{R}}
\newcommand{\regret}{\mbox{Regret}}

%%% Conditional probabilities
\newmcommandii{\condp}{p\left( #1 \left| #2 \right. \right)}

\newmcommand{\lab}{y}
\newmcommand{\ploss}{\mbox{ploss}}
\newmcommandii{\avploss}{\ploss_{#1}({#2})}
\newcommand{\sfrac}[2]{\mbox{$\frac{#1}{#2}$}}

\newcommand{\mboosta}{\mbox{\algfnt AdaBoost.M1}}
\newcommand{\mboostb}{\mbox{\algfnt AdaBoost.M2}}
\newcommand{\mboostr}{\mbox{\algfnt AdaBoost.R}}

%\newmcommand{\slos}{\mbox{ploss}}
%\newmcommandiii{\sloss}{\slos_{#1}({#2},{#3})}
%\newmcommandiii{\avsloss}{\slos_{{#1},{#2}}({#3})}

\newmcommandii{\vwt}{{W}^{#1}_{#2}}

\newcommand{\figline}{\rule{\textwidth}{1pt}}

%\newmcommandi{\1}{{\bf 1}({#1})}
\newmcommandi{\1}{[\![{#1}]\!]}

\newmcommand{\confcn}{\kappa}
\newmcommandi{\erint}{\abs{\int_{y_i}^{h_t(x_i)} {#1} dy}}
%\newmcommandi{\erint}{\int_{\min\{y_i,h_t(x_i)\}}^{\max\{y_i,h_t(x_i)\}}{#1}dy}


\begin{document}

%\iffalse %%%%%%%%%%%%%%%%%%%%%%%%%%%%%%%%%%%%%%%%%%%%%%%%%%%%%%%%%%%%%%%%%%
%\fi %%%%%%%%%%%%%%%%%%%%%%%%%%%%%%%%%%%%%%%%%%%%%%%%%%%%%%%%%%%%%%%%%%%

\begin{frame}
  \titlepage
\end{frame}

\begin{frame}
  \frametitle{Outline}
  \tableofcontents[pausesections]
  % You might wish to add the option [pausesections]
\end{frame}

\section{The Halving Algorithm}

\section{\ouralg Algorithm}

\begin{frame}
\frametitle{The hedging problem}

\begin{itemize}
\item \R{$N$} possible actions 

\item At each time step \R{$t=1,2,\ldots,T$}:
\begin{itemize}
\item Algorithm chooses a distribution \R{$\distvec{t}$} over actions.
\item Losses \R{$0 \leq \cost{t}{i} \leq 1$} of all actions \R{$i=1,\ldots,N$} are revealed.
\item Algorithm suffers {\bf expected} loss \R{$\distvec{t} \cdot \costvec{t}$}
\end{itemize}
\item {{\bf Goal:} minimize total expected loss}
\item {Here we have stochasticity - but only in {\bf algorithm}, not in {\bf outcome}}
\item {Fits nicely in game theory}
\end{itemize}
\end{frame}

\subsection{Hedging vs. Halving}

\begin{frame}
\frametitle{Hedging vs. Halving}
\begin{itemize}
\item Like halving - we want to zoom into best action (expert).
\item Unlike halving - no action is perfect.
\item Basic idea - reduce probability of lossy actions, \\
but {\color{blue}not all the way to zero}.
\item {\bf Modified Goal:}
minimize {\color{blue}{difference between}} \\
expected total loss \\
{\color{blue}{and}} \\
minimal total loss of repeating one action.
\R{\[
\sum_{t=1}^T \distvec{t} \cdot \costvec{t} - \min_i \left(\sum_{t=1}^T \cost{t}{i} \right)
\]}
\end{itemize}
\end{frame}

\begin{frame}
\frametitle{Using hedge to generalize halving alg.}
\begin{itemize}
\item Suppose that there is no perfect expert.
\item action \R{$i\;\;$} = use prediction of expert \R{$i$}
\item Now each iteration of game consistst of \R{three} steps:
\begin{itemize}
\item Experts make predictions \R{$e^t_i \in \{0,1\}$}
\item Algorithm predicts \R{$1$} with probability \R{$\sum_{i: e^t_i=1} \dist{t}{i}$}.
\item outcome \R{$o^t_i$} is revealed. \R{$\cost{t}{i}=0$} if \R{$e^t_i = o^t_i$}, \R{$\cost{t}{i}=1$} otherwise.
\end{itemize}
\end{itemize}
\end{frame}

\begin{frame}
\frametitle{The \ouralg Algorithm}
Consider action \R{$i$} at time \R{$t$}
\begin{itemize}
\item Total loss:
\R{$$L_i^t = \sum_{s=1}^{t-1} \ell_i^s$$}
\item Weight:
\R{$$\wt{t}{i} = \wt{1}{i} e^{-\eta L_i^t}$$}
Note freedom to choose initial weight (\R{$\wt{1}{i}$})
\R{$\sum_{i=1}^n \wt{1}{i} = 1$}.
\item
\R{$\eta>0$} is the learning rate parameter. Halving: \R{$\eta \to \infty$}
\item Probability:
\R{$$\dist{t}{i} = \frac{\wt{t}{i}}{\sum_{j=1}^N \wt{t}{i}},\;\;
\pause     \distvec{t} = \frac{\wtvec{t}}{\sum_{j=1}^N \wt{t}{i}}$$}
\end{itemize}
\end{frame}

\begin{frame} 
\frametitle{Choosing the initial weights} 

\begin{itemize}
\item Giving an action high initial weight makes alg perform well
  \R{if} that action performs well.
\item If good action has low initial weight, our total loss will
  be larger.
\item As \R{$\sum_{i=1}^n \wt{1}{i} = 1$} increasing one weight
  implies decreasing some others.
\item Plays a similar role to prior distribution in Bayesian
  algorithms.
\end{itemize}

\end{frame} 

\section{Bound on total loss}
\begin{frame}
\frametitle{Bound on the loss of \ouralg Algorithm}
\begin{itemize}
\item
\begin{theorem}[main theorem] \label{thm:basic-bnd}
For any sequence of loss vectors \R{$\costvec{1},\ldots,\costvec{\iter}$},
and for any \R{$i\in\{1,\ldots,N\}$}, we have
\R{\begin{equation*}
\lossouralg \leq \frac{-\ln(\wt{1}{i}) + \eta \lossi{i}}
		      {1-e^{-\eta}}.
\end{equation*}}
%% More generally, for any nonempty set $S\subseteq\{1,\ldots,N\}$, we have
%% \begin{equation}\label{eqn:set-bnd}
%% \lossouralg \leq \frac{-\ln(\sum_{i\in S}\wt{1}{i})
%% 			 - \eta \max_{i\in S} \lossi{i}}
%% 		      {1-e^{-eta}}.
%% \end{equation}
\end{theorem}
\item
\R{Proof}: by combining upper and lower bounds on \R{$\sumwts{\iter+1}$}
\end{itemize}
\end{frame}

\subsection{Upper bound on $\sumwts{\iter+1}$}

\begin{frame}
\frametitle{Upper bound on \R{$\sumwts{\iter+1}$}}
\begin{lemma}[upper bound] 
For any sequence of loss vectors \R{$\costvec{1},\ldots,\costvec{\iter}$}
we have
\R{\[
\ln\paren{\sumwts{\iter+1}} \leq -(1-e^{-\eta}) \lossouralg.
\]}
\end{lemma}
\end{frame}

\begin{frame}
\frametitle{Proof of upper bound (slide 1)}
\begin{itemize}
\item
If \R{$a \geq 0$} then \R{$a^r$} is convex.
\item For \R{$r\in [0,1]$}, \R{$a^r \leq 1-(1-a)r$}
\item
\includegraphics[height=6cm]{figures/Convexity.pdf}
\end{itemize}
\end{frame}

\begin{frame}
\frametitle{Proof of upper bound (slide 2)}
Applying \R{$a^r \leq 1-(1-a)^r$} where \R{$a=e^{-\eta}$,$r=\cost{t}{i}$}
\R{
\begin{eqnarray*}
\sumwts{t+1} &= & 
 \sum_{i=1}^N \wt{t}{i} e^{-\eta \cost{t}{i}} \\
~\pause  &\leq&  
 \sum_{i=1}^N \wt{t}{i} \left( 1-(1-e^{-\eta})\cost{t}{i}\right) \\
~\pause  &=& 
 \paren{\sumwts{t}} \paren{ 1-(1-e^{-\eta}) \frac{\wtvec{t}}{\sumwts{t}} \cdot \costvec{t}}\\
~\pause  &=& 
 \paren{\sumwts{t}} \paren{ 1-(1-e^{-\eta}) \distvec{t}\cdot\costvec{t} }
\end{eqnarray*}
}
\end{frame}

\begin{frame}
\frametitle{Proof of upper bound (slide 3)}
\begin{itemize}
\item Combining 
\R{\[
\sumwts{t+1} \leq  \paren{\sumwts{t}} \left( 1-(1-e^{-\eta}) \distvec{t}\cdot\costvec{t} \right)
\]}
\item
for \R{$t=1,\ldots,T$} 
\item yields
\R{\begin{eqnarray*}
\sumwts{T+1} &\leq& \prod_{t=1}^T (1-(1-e^{-\eta})
                         \distvec{t}\cdot\costvec{t}) \\
~\pause &\leq& \exp\paren{-(1-e^{-\eta}) \sum_{t=1}^T
                         \distvec{t}\cdot\costvec{t}} \\
\end{eqnarray*}}
since \R{$1+x\leq e^x$} for $x = -(1-e^{-\eta})$.
\end{itemize}
\end{frame}

\subsection{Lower bound on $\sumwts{\iter+1}$}

\begin{frame}
\frametitle{Lower bound on $\sumwts{\iter+1}$}

For any \R{$j=1,\ldots,N$}:
\R{\[
\sumwts{\iter+1} \geq \wt{\iter+1}{j} = \wt{1}{j} e^{-\eta \lossi{j}}
\]}

\end{frame}

\subsection{Combining Upper and Lower bounds}

\begin{frame}
\frametitle{Combining Upper and Lower bounds}
\begin{itemize}
\item
Combining bounds on \R{$\ln \paren{\sumwts{\iter+1}}$}
\R{\[
 \ln \wt{1}{j} -\eta \lossi{j} \leq \ln \sumwts{\iter+1} 
 \leq -(1-e^{-\eta}) \sum_{t=1}^T \distvec{t}\cdot\costvec{t}
\]}
\item
Reversing signs, using \R{$\lossouralg = \sum_{t=1}^T \distvec{t}\cdot\costvec{t}$} 
and reorganizing we get
\R{\[
\lossouralg \leq \frac{-\ln(\wt{1}{i}) + \eta \lossi{i}}
		      {1-e^{-\eta}}
\]}
\end{itemize}
\end{frame}

\section{tuning $\eta$}

\begin{frame}
\frametitle{Tuning \R{$\eta$}}
\includegraphics[height=7cm]{figures/beta-bounds.jpg}
\end{frame}

\begin{frame}
\frametitle{Tuning \R{$\eta$}}
\begin{itemize}
\item Suppose \R{$\min_i \lossi{i} \leq \upbnd{L}$}
\item set
\R{\[
\eta = \ln \paren{ 1+ \sqrt{\frac{2 \ln N}{\upbnd{L}}}} \approx \sqrt{\frac{2 \ln N}{\upbnd{L}}}
\]}
\item use uniform initial weights \R{$\wtvec{1} = \langle 1/N,\ldots,1/N \rangle$}
\item Then
\R{\[
\lossouralg \leq \frac{-\ln(\wt{1}{i}) + \eta \lossi{i}}
		      {1-e^{-\eta}}
\leq \min_i \lossi{i} + \sqrt{2 \upbnd{L} \ln N} + \ln N
\]}
\end{itemize}
\end{frame}

\begin{frame}
\frametitle{Tuning \R{$\eta$} as a function of \R{$T$}}
\begin{itemize}
\item trivially \R{$\min_i \lossi{i} \leq T$}, yielding
\R{\[
\lossouralg \leq \min_i \lossi{i} + \sqrt{2 T \ln N} + \ln N
\]}
\item per iteration we get:
\R{\[
\frac{\lossouralg}{T} \leq \min_i \frac{\lossi{i}}{T} + \sqrt{\frac{2 \ln N}{T}} + \frac{\ln N}{T}
\]}
\end{itemize}
\end{frame}

\section{Lower Bounds}
\begin{frame}
\frametitle{How good is this bound?}
\begin{itemize}
\item
{\color{blue} Very good!} There is a closely matching lower bound!
\item
There exists a stochastic adversarial strategy such that with high
probability for \alert{any} hedging strategy \R{${\algfnt S}$} after \R{$T$} trials
\R{\[
\lossS - \min_i \lossi{i} \geq (1-o(1)) \sqrt{2T \ln N}
\]}
\item
The adversarial strategy is random, extremely simple, and does not
depend on the hedging strategy! 
\end{itemize}
\end{frame}

\begin{frame}
\frametitle{The adversarial strategy}
\begin{itemize}
\item
Adversary sets each loss \R{$\cost{t}{i}$} indepedently at random \\
to \R{$0$} or \R{$1$} with equal probabilities $(1/2,1/2)$.
\item
Obviously, nothing to learn !\\
\R{$\lossS \approx T/2$.}
\item
On the other hand \R{$\min_i \lossi{i} \approx T/2 - \sqrt{2T \ln N}$}
\item
The difference \R{$\lossS - \min_i \lossi{i}$} is due to unlearnable
random fluctuations!
\item
Detailed proof quite involved. See games paper.
\end{itemize}
\end{frame}

\begin{frame}
\frametitle{Summary}
\begin{itemize}
\item Given learning rate \R{$\eta$} the \ouralg algorithm satisfies
\R{\[
\lossouralg \leq \frac{\ln N + \eta \lossi{i}}
		      {1-e^{-\eta}}
\]}
\item Setting \R{$\eta \approx \sqrt{\frac{2 \ln N}{T}}$} guarantees
\R{\[
\lossouralg \leq \min_i \lossi{i} + \sqrt{2 T \ln N} + \ln N
\]}
\item
A trivial random data, in which there is nothing to be learned forces
\alert{any} algorithm to suffer this total loss
\end{itemize}
\end{frame}

\begin{frame}
\frametitle{Some loose threads}
\begin{itemize}
\item Total Loss of best action usually scales linearly with time
  \R{$T$}, but we need to know the {\bf horizon} $T$ ahead of time to
  choose \R{$\eta$} correctly.
  \item \R{$T$} is tight only when the loss of experts at each iteration is
    either 0 or 1. If the loss of the best expert is $o(T)$ then we
    would like to have a tighter bound.
\item Observing only the loss of chosen action - the multi-armed
  bandit problem. Will get to that later in the course.
\item Register to the class on the google drive.
\end{itemize}
\end{frame}

\end{document}


