\documentclass[12pt]{article}
\usepackage{amsmath,amssymb, amsthm}
\usepackage{graphicx}
\usepackage{times}

\usepackage{geometry}
 \geometry{
 a4paper,
 total={170mm,257mm},
 left=20mm,
 top=20mm,
 }

\begin{document} 
 
\section{HW2 / CSE254 / 2020}

Due: Tuesday February 4.
~\\~\\

In the Halving algorithm we assume that one of the $N$ experts never makes
a mistake. Suppose instead the assumption is that one of the experts
makes at most one mistake. What is the best way to predict in this
case?
~\\
~\\
~\\
{\bf Hints}
\begin{itemize}
  \item Suppose you have an upper bound of $M$ on the number of
    mistakes of your yet-to-be-born algorithm. Given that assumption,
    can you reduce the problem to the old problem, where one of the
    experts makes no mistake.
  \item You will have to increase the number of experts to do this reduction.
  \item Once you have the reduction, use the $\log_2 N$ bound on the
    halving algorithm to get a bound on your algorithm.
  \item You have the main pieces, and algorithm and  a bound. Next you
    have to work on keeping everything consistent.
  \item There is a version of this algorithm where everything is
    planned in advance and you assume that the adversary is playing
    optimally. There is a more sophisticated algorithm that takes
    advantage of the mistakes of a suboptimal adversary.
\end{itemize}

\end{document}
